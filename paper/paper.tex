%% bare_jrnl.tex
%% V1.3
%% 2007/01/11
%% by Michael Shell
%% see http://www.michaelshell.org/
%% for current contact information.
%%
%% This is a skeleton file demonstrating the use of IEEEtran.cls
%% (requires IEEEtran.cls version 1.7 or later) with an IEEE journal paper.
%%
%% Support sites:
%% http://www.michaelshell.org/tex/ieeetran/
%% http://www.ctan.org/tex-archive/macros/latex/contrib/IEEEtran/
%% and
%% http://www.ieee.org/



% *** Authors should verify (and, if needed, correct) their LaTeX system  ***
% *** with the testflow diagnostic prior to trusting their LaTeX platform ***
% *** with production work. IEEE's font choices can trigger bugs that do  ***
% *** not appear when using other class files.                            ***
% The testflow support page is at:
% http://www.michaelshell.org/tex/testflow/


%%*************************************************************************
%% Legal Notice:
%% This code is offered as-is without any warranty either expressed or
%% implied; without even the implied warranty of MERCHANTABILITY or
%% FITNESS FOR A PARTICULAR PURPOSE! 
%% User assumes all risk.
%% In no event shall IEEE or any contributor to this code be liable for
%% any damages or losses, including, but not limited to, incidental,
%% consequential, or any other damages, resulting from the use or misuse
%% of any information contained here.
%%
%% All comments are the opinions of their respective authors and are not
%% necessarily endorsed by the IEEE.
%%
%% This work is distributed under the LaTeX Project Public License (LPPL)
%% ( http://www.latex-project.org/ ) version 1.3, and may be freely used,
%% distributed and modified. A copy of the LPPL, version 1.3, is included
%% in the base LaTeX documentation of all distributions of LaTeX released
%% 2003/12/01 or later.
%% Retain all contribution notices and credits.
%% ** Modified files should be clearly indicated as such, including  **
%% ** renaming them and changing author support contact information. **
%%
%% File list of work: IEEEtran.cls, IEEEtran_HOWTO.pdf, bare_adv.tex,
%%                    bare_conf.tex, bare_jrnl.tex, bare_jrnl_compsoc.tex
%%*************************************************************************

% Note that the a4paper option is mainly intended so that authors in
% countries using A4 can easily print to A4 and see how their papers will
% look in print - the typesetting of the document will not typically be
% affected with changes in paper size (but the bottom and side margins will).
% Use the testflow package mentioned above to verify correct handling of
% both paper sizes by the user's LaTeX system.
%
% Also note that the "draftcls" or "draftclsnofoot", not "draft", option
% should be used if it is desired that the figures are to be displayed in
% draft mode.
%
\documentclass[journal]{IEEEtran}
%
% If IEEEtran.cls has not been installed into the LaTeX system files,
% manually specify the path to it like:
% \documentclass[journal]{../sty/IEEEtran}





% Some very useful LaTeX packages include:
% (uncomment the ones you want to load)


% *** MISC UTILITY PACKAGES ***
%
%\usepackage{ifpdf}
% Heiko Oberdiek's ifpdf.sty is very useful if you need conditional
% compilation based on whether the output is pdf or dvi.
% usage:
% \ifpdf
%   % pdf code
% \else
%   % dvi code
% \fi
% The latest version of ifpdf.sty can be obtained from:
% http://www.ctan.org/tex-archive/macros/latex/contrib/oberdiek/
% Also, note that IEEEtran.cls V1.7 and later provides a builtin
% \ifCLASSINFOpdf conditional that works the same way.
% When switching from latex to pdflatex and vice-versa, the compiler may
% have to be run twice to clear warning/error messages.






% *** CITATION PACKAGES ***
%
%\usepackage{cite}
% cite.sty was written by Donald Arseneau
% V1.6 and later of IEEEtran pre-defines the format of the cite.sty package
% \cite{} output to follow that of IEEE. Loading the cite package will
% result in citation numbers being automatically sorted and properly
% "compressed/ranged". e.g., [1], [9], [2], [7], [5], [6] without using
% cite.sty will become [1], [2], [5]--[7], [9] using cite.sty. cite.sty's
% \cite will automatically add leading space, if needed. Use cite.sty's
% noadjust option (cite.sty V3.8 and later) if you want to turn this off.
% cite.sty is already installed on most LaTeX systems. Be sure and use
% version 4.0 (2003-05-27) and later if using hyperref.sty. cite.sty does
% not currently provide for hyperlinked citations.
% The latest version can be obtained at:
% http://www.ctan.org/tex-archive/macros/latex/contrib/cite/
% The documentation is contained in the cite.sty file itself.






% *** GRAPHICS RELATED PACKAGES ***
%
\ifCLASSINFOpdf
  % \usepackage[pdftex]{graphicx}
  % declare the path(s) where your graphic files are
  % \graphicspath{{../pdf/}{../jpeg/}}
  % and their extensions so you won't have to specify these with
  % every instance of \includegraphics
  % \DeclareGraphicsExtensions{.pdf,.jpeg,.png}
\else
  % or other class option (dvipsone, dvipdf, if not using dvips). graphicx
  % will default to the driver specified in the system graphics.cfg if no
  % driver is specified.
  % \usepackage[dvips]{graphicx}
  % declare the path(s) where your graphic files are
  % \graphicspath{{../eps/}}
  % and their extensions so you won't have to specify these with
  % every instance of \includegraphics
  % \DeclareGraphicsExtensions{.eps}
\fi
% graphicx was written by David Carlisle and Sebastian Rahtz. It is
% required if you want graphics, photos, etc. graphicx.sty is already
% installed on most LaTeX systems. The latest version and documentation can
% be obtained at: 
% http://www.ctan.org/tex-archive/macros/latex/required/graphics/
% Another good source of documentation is "Using Imported Graphics in
% LaTeX2e" by Keith Reckdahl which can be found as epslatex.ps or
% epslatex.pdf at: http://www.ctan.org/tex-archive/info/
%
% latex, and pdflatex in dvi mode, support graphics in encapsulated
% postscript (.eps) format. pdflatex in pdf mode supports graphics
% in .pdf, .jpeg, .png and .mps (metapost) formats. Users should ensure
% that all non-photo figures use a vector format (.eps, .pdf, .mps) and
% not a bitmapped formats (.jpeg, .png). IEEE frowns on bitmapped formats
% which can result in "jaggedy"/blurry rendering of lines and letters as
% well as large increases in file sizes.
%
% You can find documentation about the pdfTeX application at:
% http://www.tug.org/applications/pdftex





% *** MATH PACKAGES ***
%
%\usepackage[cmex10]{amsmath}
% A popular package from the American Mathematical Society that provides
% many useful and powerful commands for dealing with mathematics. If using
% it, be sure to load this package with the cmex10 option to ensure that
% only type 1 fonts will utilized at all point sizes. Without this option,
% it is possible that some math symbols, particularly those within
% footnotes, will be rendered in bitmap form which will result in a
% document that can not be IEEE Xplore compliant!
%
% Also, note that the amsmath package sets \interdisplaylinepenalty to 10000
% thus preventing page breaks from occurring within multiline equations. Use:
%\interdisplaylinepenalty=2500
% after loading amsmath to restore such page breaks as IEEEtran.cls normally
% does. amsmath.sty is already installed on most LaTeX systems. The latest
% version and documentation can be obtained at:
% http://www.ctan.org/tex-archive/macros/latex/required/amslatex/math/





% *** SPECIALIZED LIST PACKAGES ***
%
%\usepackage{algorithmic}
% algorithmic.sty was written by Peter Williams and Rogerio Brito.
% This package provides an algorithmic environment fo describing algorithms.
% You can use the algorithmic environment in-text or within a figure
% environment to provide for a floating algorithm. Do NOT use the algorithm
% floating environment provided by algorithm.sty (by the same authors) or
% algorithm2e.sty (by Christophe Fiorio) as IEEE does not use dedicated
% algorithm float types and packages that provide these will not provide
% correct IEEE style captions. The latest version and documentation of
% algorithmic.sty can be obtained at:
% http://www.ctan.org/tex-archive/macros/latex/contrib/algorithms/
% There is also a support site at:
% http://algorithms.berlios.de/index.html
% Also of interest may be the (relatively newer and more customizable)
% algorithmicx.sty package by Szasz Janos:
% http://www.ctan.org/tex-archive/macros/latex/contrib/algorithmicx/




% *** ALIGNMENT PACKAGES ***
%
%\usepackage{array}
% Frank Mittelbach's and David Carlisle's array.sty patches and improves
% the standard LaTeX2e array and tabular environments to provide better
% appearance and additional user controls. As the default LaTeX2e table
% generation code is lacking to the point of almost being broken with
% respect to the quality of the end results, all users are strongly
% advised to use an enhanced (at the very least that provided by array.sty)
% set of table tools. array.sty is already installed on most systems. The
% latest version and documentation can be obtained at:
% http://www.ctan.org/tex-archive/macros/latex/required/tools/


%\usepackage{mdwmath}
%\usepackage{mdwtab}
% Also highly recommended is Mark Wooding's extremely powerful MDW tools,
% especially mdwmath.sty and mdwtab.sty which are used to format equations
% and tables, respectively. The MDWtools set is already installed on most
% LaTeX systems. The lastest version and documentation is available at:
% http://www.ctan.org/tex-archive/macros/latex/contrib/mdwtools/


% IEEEtran contains the IEEEeqnarray family of commands that can be used to
% generate multiline equations as well as matrices, tables, etc., of high
% quality.


%\usepackage{eqparbox}
% Also of notable interest is Scott Pakin's eqparbox package for creating
% (automatically sized) equal width boxes - aka "natural width parboxes".
% Available at:
% http://www.ctan.org/tex-archive/macros/latex/contrib/eqparbox/





% *** SUBFIGURE PACKAGES ***
%\usepackage[tight,footnotesize]{subfigure}
% subfigure.sty was written by Steven Douglas Cochran. This package makes it
% easy to put subfigures in your figures. e.g., "Figure 1a and 1b". For IEEE
% work, it is a good idea to load it with the tight package option to reduce
% the amount of white space around the subfigures. subfigure.sty is already
% installed on most LaTeX systems. The latest version and documentation can
% be obtained at:
% http://www.ctan.org/tex-archive/obsolete/macros/latex/contrib/subfigure/
% subfigure.sty has been superceeded by subfig.sty.



%\usepackage[caption=false]{caption}
%\usepackage[font=footnotesize]{subfig}
% subfig.sty, also written by Steven Douglas Cochran, is the modern
% replacement for subfigure.sty. However, subfig.sty requires and
% automatically loads Axel Sommerfeldt's caption.sty which will override
% IEEEtran.cls handling of captions and this will result in nonIEEE style
% figure/table captions. To prevent this problem, be sure and preload
% caption.sty with its "caption=false" package option. This is will preserve
% IEEEtran.cls handing of captions. Version 1.3 (2005/06/28) and later 
% (recommended due to many improvements over 1.2) of subfig.sty supports
% the caption=false option directly:
%\usepackage[caption=false,font=footnotesize]{subfig}
%
% The latest version and documentation can be obtained at:
% http://www.ctan.org/tex-archive/macros/latex/contrib/subfig/
% The latest version and documentation of caption.sty can be obtained at:
% http://www.ctan.org/tex-archive/macros/latex/contrib/caption/




% *** FLOAT PACKAGES ***
%
%\usepackage{fixltx2e}
% fixltx2e, the successor to the earlier fix2col.sty, was written by
% Frank Mittelbach and David Carlisle. This package corrects a few problems
% in the LaTeX2e kernel, the most notable of which is that in current
% LaTeX2e releases, the ordering of single and double column floats is not
% guaranteed to be preserved. Thus, an unpatched LaTeX2e can allow a
% single column figure to be placed prior to an earlier double column
% figure. The latest version and documentation can be found at:
% http://www.ctan.org/tex-archive/macros/latex/base/



%\usepackage{stfloats}
% stfloats.sty was written by Sigitas Tolusis. This package gives LaTeX2e
% the ability to do double column floats at the bottom of the page as well
% as the top. (e.g., "\begin{figure*}[!b]" is not normally possible in
% LaTeX2e). It also provides a command:
%\fnbelowfloat
% to enable the placement of footnotes below bottom floats (the standard
% LaTeX2e kernel puts them above bottom floats). This is an invasive package
% which rewrites many portions of the LaTeX2e float routines. It may not work
% with other packages that modify the LaTeX2e float routines. The latest
% version and documentation can be obtained at:
% http://www.ctan.org/tex-archive/macros/latex/contrib/sttools/
% Documentation is contained in the stfloats.sty comments as well as in the
% presfull.pdf file. Do not use the stfloats baselinefloat ability as IEEE
% does not allow \baselineskip to stretch. Authors submitting work to the
% IEEE should note that IEEE rarely uses double column equations and
% that authors should try to avoid such use. Do not be tempted to use the
% cuted.sty or midfloat.sty packages (also by Sigitas Tolusis) as IEEE does
% not format its papers in such ways.


%\ifCLASSOPTIONcaptionsoff
%  \usepackage[nomarkers]{endfloat}
% \let\MYoriglatexcaption\caption
% \renewcommand{\caption}[2][\relax]{\MYoriglatexcaption[#2]{#2}}
%\fi
% endfloat.sty was written by James Darrell McCauley and Jeff Goldberg.
% This package may be useful when used in conjunction with IEEEtran.cls'
% captionsoff option. Some IEEE journals/societies require that submissions
% have lists of figures/tables at the end of the paper and that
% figures/tables without any captions are placed on a page by themselves at
% the end of the document. If needed, the draftcls IEEEtran class option or
% \CLASSINPUTbaselinestretch interface can be used to increase the line
% spacing as well. Be sure and use the nomarkers option of endfloat to
% prevent endfloat from "marking" where the figures would have been placed
% in the text. The two hack lines of code above are a slight modification of
% that suggested by in the endfloat docs (section 8.3.1) to ensure that
% the full captions always appear in the list of figures/tables - even if
% the user used the short optional argument of \caption[]{}.
% IEEE papers do not typically make use of \caption[]'s optional argument,
% so this should not be an issue. A similar trick can be used to disable
% captions of packages such as subfig.sty that lack options to turn off
% the subcaptions:
% For subfig.sty:
% \let\MYorigsubfloat\subfloat
% \renewcommand{\subfloat}[2][\relax]{\MYorigsubfloat[]{#2}}
% For subfigure.sty:
% \let\MYorigsubfigure\subfigure
% \renewcommand{\subfigure}[2][\relax]{\MYorigsubfigure[]{#2}}
% However, the above trick will not work if both optional arguments of
% the \subfloat/subfig command are used. Furthermore, there needs to be a
% description of each subfigure *somewhere* and endfloat does not add
% subfigure captions to its list of figures. Thus, the best approach is to
% avoid the use of subfigure captions (many IEEE journals avoid them anyway)
% and instead reference/explain all the subfigures within the main caption.
% The latest version of endfloat.sty and its documentation can obtained at:
% http://www.ctan.org/tex-archive/macros/latex/contrib/endfloat/
%
% The IEEEtran \ifCLASSOPTIONcaptionsoff conditional can also be used
% later in the document, say, to conditionally put the References on a 
% page by themselves.





% *** PDF, URL AND HYPERLINK PACKAGES ***
%
%\usepackage{url}
% url.sty was written by Donald Arseneau. It provides better support for
% handling and breaking URLs. url.sty is already installed on most LaTeX
% systems. The latest version can be obtained at:
% http://www.ctan.org/tex-archive/macros/latex/contrib/misc/
% Read the url.sty source comments for usage information. Basically,
% \url{my_url_here}.





% *** Do not adjust lengths that control margins, column widths, etc. ***
% *** Do not use packages that alter fonts (such as pslatex).         ***
% There should be no need to do such things with IEEEtran.cls V1.6 and later.
% (Unless specifically asked to do so by the journal or conference you plan
% to submit to, of course. )


% correct bad hyphenation here
\hyphenation{op-tical net-works semi-conduc-tor ordering algorithms}


\usepackage{subfigure,graphicx,url,times,multirow,amsmath,amssymb,algorithm,xspace,epsfig,todonotes,array,caption,color}
\usepackage{algorithm, mathtools}
\usepackage[noend]{algpseudocode}
\usepackage[bookmarks=false]{hyperref}
\hyphenation{op-tical net-works semi-conduc-tor IEEEtran instead}

\newcommand{\degree}{^{\circ}}
\newcommand{\set}[1]{\left\{#1\right\}}
\newcommand{\setb}[2]{\set{#1 \ : \ #2}}
\newcommand{\argmax}{\operatorname*{arg\,max}}

\newcommand{\twoIndent}{\hskip\algorithmicindent\hskip\algorithmicindent}
\newcommand{\threeIndent}{\hskip\algorithmicindent\hskip\algorithmicindent\hskip\algorithmicindent}

\begin{document}
%
% paper title
% can use linebreaks \\ within to get better formatting as desired


\title{\ \\ \LARGE\bf Multi-Objective Monte Carlo Tree Search for Real-Time Games \thanks{Diego~Perez, Spyridon~Samothrakis, Simon~M.~Lucas (School of Computer Science and Electronic Engineering, University of Essex, Colchester CO4 3SQ, UK; email: {\tt \{dperez,ssamot,sml\}@essex.ac.uk}); \newline
Sanaz~Mostaghim (Department of Knowledge and Language Engineering, Otto-von-Guericke-Universität Magdeburg, Magdeburg, Germany; email: {\tt sanaz.mostaghim@ovgu.de})}}

%
%
% author names and IEEE memberships
% note positions of commas and nonbreaking spaces ( ~ ) LaTeX will not break
% a structure at a ~ so this keeps an author's name from being broken across
% two lines.
% use \thanks{} to gain access to the first footnote area
% a separate \thanks must be used for each paragraph as LaTeX2e's \thanks
% was not built to handle multiple paragraphs
%

\author{Diego Perez, {\it Student Member, IEEE}, Sanaz Mostaghim, Spyridon Samothrakis, {\it Student Member, IEEE}, Simon M. Lucas, {\it Senior Member, IEEE}}

% note the % following the last \IEEEmembership and also \thanks - 
% these prevent an unwanted space from occurring between the last author name
% and the end of the author line. i.e., if you had this:
% 
% \author{....lastname \thanks{...} \thanks{...} }
%                     ^------------^------------^----Do not want these spaces!
%
% a space would be appended to the last name and could cause every name on that
% line to be shifted left slightly. This is one of those "LaTeX things". For
% instance, "\textbf{A} \textbf{B}" will typeset as "A B" not "AB". To get
% "AB" then you have to do: "\textbf{A}\textbf{B}"
% \thanks is no different in this regard, so shield the last } of each \thanks
% that ends a line with a % and do not let a space in before the next \thanks.
% Spaces after \IEEEmembership other than the last one are OK (and needed) as
% you are supposed to have spaces between the names. For what it is worth,
% this is a minor point as most people would not even notice if the said evil
% space somehow managed to creep in.



% The paper headers
\markboth{IEEE Transactions on Computational Intelligence and AI in Games}%
{Shell \MakeLowercase{\textit{et al.}}: Bare Demo of IEEEtran.cls for Journals}

% The only time the second header will appear is for the odd numbered pages
% after the title page when using the twoside option.
% 
% *** Note that you probably will NOT want to include the author's ***
% *** name in the headers of peer review papers.                   ***
% You can use \ifCLASSOPTIONpeerreview for conditional compilation here if
% you desire.




% If you want to put a publisher's ID mark on the page you can do it like
% this:
%\IEEEpubid{0000--0000/00\$00.00~\copyright~2007 IEEE}
% Remember, if you use this you must call \IEEEpubidadjcol in the second
% column for its text to clear the IEEEpubid mark.



% use for special paper notices
%\IEEEspecialpapernotice{(Invited Paper)}




\maketitle

\begin{abstract}
Abstract... 
\end{abstract}


% IEEEtran.cls defaults to using nonbold math in the Abstract.
% This preserves the distinction between vectors and scalars. However,
% if the journal you are submitting to favors bold math in the abstract,
% then you can use LaTeX's standard command \boldmath at the very start
% of the abstract to achieve this. Many IEEE journals frown on math
% in the abstract anyway.



% For peer review papers, you can put extra information on the cover
% page as needed:
% \ifCLASSOPTIONpeerreview
% \begin{center} \bfseries EDICS Category: 3-BBND \end{center}
% \fi
%
% For peerreview papers, this IEEEtran command inserts a page break and
% creates the second title. It will be ignored for other modes.
\IEEEpeerreviewmaketitle


%\IEEEoverridecommandlockouts
%
%\textwidth 178mm
%\textheight 239mm
%\oddsidemargin -7mm
%\evensidemargin -7mm
%\topmargin -6mm
%\columnsep 5mm


\section{Introduction} \label{sec:intro}

Here it goes, the introduction.

\section{Monte Carlo Tree Search} \label{sec:mcts}

Monte Carlo Tree Search (MCTS) is a tree search algorithm that was originally applied to board games, concretely to the two-players game of Go. This game is played in a square grid board, with a size of $19 \times 19$ in the original game, and $9 \times 9$ in its reduced version. The game is played in turns, and the objective is to surround the opponent's stones by placing stones in any available position in the board. Due to the very large branching factor of Go, this game is considered the drosophila of Game AI, and MCTS players have reached professional level play in the reduced board size version~\cite{Lee2009}. After its success in Go, MCTS has been used extensively by many researchers in this and different domains. An extensive survey of MCTS methods, variations and applications, has been written by Browne et al.~\cite{Browne2012}. 

MCTS is considered to be an \textit{anytime} algorithm, as it is able to provide a a valid next move to choose at any moment in time. This is true independently from how many iterations the algorithm was able to make (although, in general, more iterations usually produce better results). This differs from other algorithms (such A*) that normally provide the next ply only after they have finished. This makes MCTS a suitable candidate for real-time domains, where the decision time budget is limited, affecting the number of iterations that can be performed.

MCTS is an algorithm that builds a tree in memory. Each node in the tree maintains statistics that indicate how often a move is played from a given state ($N(s,a)$), how many times each move is played from there ($N(s)$) and the average reward ($Q(s,a)$) obtained after applying move $a$ in state $s$. The tree is built iteratively by simulating actions in the game, making move choices based on the statistics store in the nodes. 

Each iteration of MCTS can be divided into several steps~\cite{Gelly2006}: \textit{Tree selection}, \textit{Expansion}, \textit{Monte Carlo simulation} and \textit{Back-propagation} (all summarized in Figure~\ref{fig:mcts}). When the algorithm starts, the tree is formed only by the root node, which holds the current state of the game. During the \textit{selection} step, the tree is navigated from the root until a maximum depth or the end of the game has been reached. 

\begin{figure} [!t]
	\begin{center}
	\includegraphics[scale=0.4,natwidth=708,natheight=484]{img/mcts.png}
	\caption{MCTS algorithm steps.}
	\label{fig:mcts}
	\end{center}
\end{figure}

In every one of this action decisions, MCTS balances between exploitation and exploration. In other words, this chooses between taking an action that leads to states with the best outcome found so far, and performing a move to go to less explored game states, respectively. In order to achieve this, MCTS uses Upper Confidence Bound (UCB1, see Equation~\ref{eq:ucb1}) as a \textit{Tree Policy}. 

\begin{equation}	\label{eq:ucb1}
a^* = \argmax_{a \in A(s)} \left\{Q(s,a) + C \sqrt{\frac{ \ln N(s) }{ N(s,a) }}\right\}
\end{equation}

The balance between exploration and exploitation is achieved by setting the value of $C$. Higher values of $C$ weight more the second term of the UCB1 Equation~\ref{eq:ucb1}, giving preference to those actions that have been explored less, at the expense of taking actions with the highest average reward $Q(s,a)$. A commonly used value is $\sqrt{2}$, as it balances both facets of the search when the rewards are normalized between $0$ and $1$. It is worth noting that MCTS, when combined with UCB1 reaches asymptotically logarithmic regret~\cite{coquelin2007bandit}. 

If, during the \textit{tree selection} phase, a node has less children than the available number of actions from a given position, a new node is added as a child of the current one (\textit{expansion} phase) and the \textit{simulation} step starts. At this point, MCTS executes a Monte Carlo simulation (or roll-out; \textit{default policy}) from the expanded node. This is performed by choosing random (either uniformly random, or biased) actions until the game end or a pre-defined depth is reached, where the state of the game is evaluated. 

Finally, during the \textit{back-propagation} step, the statistics $N(s)$, $N(s,a)$ and $Q(s,a)$ are updated for each node visited, using the reward obtained in the evaluation of the state. These steps are executed in a loop until a termination criteria is met (such as number of iterations).


MCTS has been employed extensively in real-time games in the literature. A clear example of this is the popular real-time game \textit{Ms. PacMan}. The objective of this game is to control Ms. PacMan to clear the maze by eating all pills, without being captured by the ghosts. An important feature of this game is that it is \textit{open-ended}, as an end game situation is, most of the time, far ahead in the future and can not be devised by the algorithm during its iterations. The consequence of this is that MCTS, in its vanilla form, it is not able to know if a given ply will lead to a win or a loss game end state. Robles et al~\cite{Robles2009} solved this problem by including hand-coded heuristics that guided MCTS simulations towards more promising portions of the search space. Other authors also included domain knowledge to bias the search in MCTS, such as in~\cite{Samothrakis, Ikehata2010}.

MCTS has also been applied to single-player games, like SameGame~\cite{Matsumoto2010}, where the player's goal is to destroy contiguous tiles of the same colour, distributed in a rectangular grid. Another use of MCTS is in the popular puzzle Morpion Solitaire~\cite{Edelkamp2010}, a connection game where the goal is to link nodes of a graph with straight lines that must contain at least five vertices. Finally, the PTSP has also been addressed with MCTS, both in the single-objective~\cite{Perez2013, Powley2012} and the multi-objective versions~\cite{Powley2013}. These papers describe the entries that won both editions of the PTSP Competition.

It is worthwhile mentioning that in most cases found in the literature, MCTS techniques have been used with some kind of heuristic that guides the Monte Carlo simulations or the tree selection policy. In the algorithm proposed in this paper, simulations are purely random, as the objective is to compare the search abilities of the different algorithms. The intention is therefore to keep the heuristics to a minimum, and the existing pieces of domain knowledge are shared by all the algorithms presented (as in the case of the score function for MO-PTSP, described later).


\section{Multi-Objective Optimization} \label{sec:moo}

A multi-objective optimization problem (MOOP) represents a scenario where two or more objective functions are to be optimized (either maximized or minimized). The general form of a MOOP is formally described as a maximization function $F_m(x)$, that transforms points in the decision space ($X$) to points in the solution space ($F$). The elements of the decision space are vectors of $n$ variables of the form $x = (x_1, x_2, \dots, x_n)$, while elements in the solution space are vectors with a dimension $m$: $F_m(x) = (f_1(x), f_2(x), \dots, f_m(x))$. Therefore, each solution provides $m$ different scores (or rewards, or fitness) that are meant to be optimized. Without loss of generality, it is assumed from now on that all objectives must be maximized.

It is said that a solution $F_m(x)$ \textit{dominates} another solution $F_m(y)$ if:

\begin{enumerate}
\item $F_m(x)$ is not worse than $F_m(y)$ in all objectives for all $i = 1, 2, \dots, m$.
\item At least one objective of $F_m(x)$ is better than its analogous counterpart in $F_m(y)$.
\end{enumerate}


When this two conditions apply, it is said that $F_m(x) \preceq F_m(y)$ ($F_m(x)$ dominates $F_m(y)$, and $F_m(x)$ is non-dominated by $F_m(y)$). The \textit{dominance} condition provides a partial ordering between points in the solution space: if $F_m(x) \preceq F_m(y)$, then $F_m(x)$ is considered to be better than $F_m(y)$.

However, there are some cases where it cannot be said that $F_m(x) \preceq F_m(y)$ or $F_m(y) \preceq F_m(x)$. This situation occurs when one of the objectives is better in $F_m(x)$ but a different objective is better in $F_m(y)$ (for instance, when $F_1(x) < F_1(y)$ but $F_2(x) > F_2(y)$). In this case, it is said that these solutions are non-dominated with respect to each other. Solutions that are not dominated by each other are grouped in a \textit{non-dominated set}. Given a non-dominated set $P$, it is said that $P$ is the \textit{optimal Pareto front} if there is no other solution in the solution space that dominates any member of $P$. The relation between decision and objective space, dominance and Pareto fronts is depicted in Figure~\ref{fig:moop}.

\begin{figure} [!t]
	\begin{center}
	\includegraphics[scale=0.235]{img/moopDec.png}
	\caption{Decision and Solution spaces in a MOOP with two variables ($x_1$ and $x_2$) and two objectives ($f_1$ and $f_2$). In the objective space, yellow dots are non-optimal objective vectors, while blue dots form the Pareto-optimal front.}
	\label{fig:moop}
	\end{center}
\end{figure}

As many Pareto front can be formed by several points in the solution space, it is important to devise a mechanism to assess the quality of a given front. A possibility is to use the Hypervolume Indicator (HV): given a Pareto front $P$, $HV(P)$ is defined as the volume of the objective space dominated by $P$. More formally, $HV(P) = \mu (x \in \mathbb{R}^{d} : \exists r \in P \hspace{0.25cm} s.t \hspace{0.25cm} r \preceq x)$, where $\mu$ is the de Lebesgue measure on $\mathbb{R}^{d}$. If the objectives are to be maximized, the higher the $HV(P)$, the better the front calculated. Figure~\ref{fig:hv} shows an example of $HV(P)$ where the objective dimension space is $2$.

\begin{figure}[!t]
\begin{center}
\includegraphics[width=0.75\columnwidth]{img/hv}
\end{center}
\caption{$HV(P)$ of a given pareto front $P$.}
\label{fig:hv}
\end{figure}

For more extensive descriptions, definitions, properties and multi-objective optimization in general, the reader can consult the work by K. Deb~\cite{Deb2001}.

Many different algorithms have been proposed to tackle multi-objective optimization problems in the literature. One of the most widely known and used methods is the weighted-sum approach. The procedure consists of giving a weight to each one of the objective and produce a single result as the linear combination of objectives and weights. By varying the weights provided, it is possible to converge to different solutions of the optimal Pareto front, if this is convex. However, K. Deb~\cite{Deb2001} explains how linear scalarization approaches fail in those scenarios where the optimal Pareto front is non-convex.

A popular choice for multi-objective optimization problems are evolutionary multi-objective optimization (EMOA) algorithms~\cite{Coello1999, Zhou2011}. One of the most well known algorithms in the literature is the Non-dominated Sorting Evolutionary Algorithm 2 (NSGA-2), which pseudocode is shown in Algorithm~\ref{alg:nsga2}. As in any evolutionary algorithm, NSGA-2 evolves a set of individuals or solutions to the problem, with the difference that here they are ranked according to dominance criteria and crowding distance (distances between members of the pareto fronts). A full description of the algorithm can be found in~\cite{Deb2000}.

\begin{algorithm}[!t]
\begin{algorithmic}[1]
\Function{NSGA-2}{}
	\State $P = NewRandomPopulation$
	\While{Termination criteria not met}
		\State $R = P \cup Q$ 
		\State $F = \Call{fastNonDominatedSort}{R}$
		\While{$|P| < N$}
			\State $\Call{crowdingDistanceAssignment}{F_i}$	
			\State $P = P \cup F_i$
		\EndWhile
		\State $\Call{sort}{P}$
		\State $P = P[0:N]$
		\State $Q = breed(P)$	
	\EndWhile
\EndFunction
\end{algorithmic}
\caption{NSGA-2 Algorithm.}
\label{alg:nsga2}
\end{algorithm}

The three main pillars of the NSGA-2 algorithm are:

\begin{itemize}
\item A \textit{fast non-dominated sorting} algorithm, that ranks the individuals of the population and groups them in Pareto fronts.
\item A \textit{crowding distance}, assigned to each one of the individuals, that measures how close it is to its neighbours. The selection genetic operator chooses individuals based on the ranks of the individuals and their crowding distance.
\item \textit{Elitism}, implemented so the algorithm automatically promotes the best $N$ individuals to the next generation.
\end{itemize}

A more recent approach, developed by Q. Zhang, is the Multi-Objective Evolutionary Algorithm based on Decomposition (MOEA/D)~\cite{Zhang2007}, that decomposes the problem into several single optimization sub-problems and an evolutionary algorithm optimizes them all simultaneously. Information is shared between neighbouring sub-problems in order to guide evolution. The authors show that MOEA/D performs similarly than, and sometimes even outperforms, NSGA-II in the scenarios tested.

Reinforcement Learning (RL) algorithms have also tackled Multi-objective optimization in some scenarios. RL~\cite{Sutton1998} is a broad field in Machine Learning that studies real-time planning and control situations where  an agent has to find out the actions (or sequences of actions) that should be applied in order to maximize the reward from the environment. 

An RL problem can be defined as a tuple ($S$, $A$, $T$, $R$, $\pi$). $S$ is the set of possible states in the problem (or game), and $s_0$ is the initial state. $A$ is the set of available actions the agent can make at any given time, and the transition model $T(s_i, a_i, s_{i+1})$ determines the probability of reaching the state $s_{i+1}$ when action $a_i$ is applied in state $s_i$. The reward function $R(s_i)$ provides a single value (\textit{reward}) that the agent must optimize, representing the desirability of the state $s_i$ reached. Finally, a decision policy $\pi(s_i) = a_i$ determines which actions $a_i$ must be chosen from each state $s_i \in S$. One of the most important challenges in RL, as shown in Section~\ref{sec:mcts}, is the trade-off between exploration and exploitation. The decision policy can choose between following actions that provided good rewards in the past and exploring new parts of the search space by selecting new actions. 

Multi-objective Reinforcement Learning (MORL)~\cite{Vamplew2010} changes this definition by using instead a vector $R = {r_0, r_1, \dots, r_n}$ as rewards of the problem. Thus, MORL problems differ from RL in having more than one objective objectives ($n$) that must be maximized. If the objectives are independent or they do not oppose each other, scalarization technique approaches, as described above, could be suitable to tackle the problem. Essentially, this would mean to use a conventional RL algorithm on a single objective obtained by a weighted-sum of the multiple rewards. However, this is not always the case, as it is usual that the objectives are in conflict and the policy ($\pi$) must balance among them.

According to how $\pi$ approaches this problem, Vamplew et al.~\cite{Vamplew2010} proposed the following distinction for MORL: \textit{single-policy} algorithms are those that provide a preference order in the objectives available (given by the user or by the nature of the problem). An example of this type of algorithm can be found at~\cite{Gabor1998}, where the authors introduce an order of preference in the objectives treated and constraint the value of the rewards desired. Scalarization approaches would also fit in this category, as the work performed by S. Natarajan et al.~\cite{Natarajan2005}.

The second type of algorithms, \textit{multiple-policy}, target to approximate the optimal Pareto front of the problem. An example of this type of algorithm is the one given by L. Barrett~\cite{Barrett2008}, who propose the Convex Hull Iteration Algorithm. This algorithm provides the optimal policy for any linear preference function, by learning all policies that define the convex hull of the Pareto front.

\section{Multi-Objective Monte Carlo Tree Search} \label{sec:momcts}

Adapting MCTS into Multi-Objective Monte Carlo Tree Search (MO-MCTS) requires the obvious modification of dealing with multiple rewards instead of just one. As these are collected at the end of a Monte Carlo simulation, the reward value $r$ now becomes a vector $R = {r_0, r_1, \dots, r_n}$, where $n$ is the number of objectives to optimize. Derived from this change, the average value $Q(s,a)$ becomes a vector that stores the average reward of each objective. Note that the other statistics ($N(s,a)$ and $N(s)$) do not need to change, as this are just node and action counters. The important question to answer next is how to adapt the vector $Q(s,a)$ to use it in the UCB1 formula (Equation~\ref{eq:ucb1}). 

An initial attempt on Multi-Objective MCTS was addressed by W. Wang and Michele Sebag~\cite{Weijia2012, Weijia2013}. In their work, the authors employ a mechanism, based on the HV calculation, to replace the UCB1 equation. The algorithm keeps a Pareto archive (P) with the best solutions found in game end states. Every node in the tree defines $\overline{r}_{sa}$ as a vector of UCB1 values, in which each $\overline{r}_{sa, i}$ is the result of calculating UCB1 for each particular objective $i$.

The next step is to define the value for each pair of state and action, $W(s,a)$, as in Equation~\ref{eq:wang}. $\overline{r}^p_{sa}$ is the projection of $\overline{r}_{sa}$ into the piecewise linear surface defined by the Pareto archive $P$ (see Figure~\ref{fig:wangHV}. Then, $HV(P \cup \overline{r}_{sa})$ is declared as the HV of $P$ plus the point $\overline{r}_{sa}$. If $\overline{r}_{sa}$ is dominated by P, the distance between $\overline{r}_{sa}$ and $\overline{r}^p_{sa}$ is subtracted from the HV calculation. The tree policy selects actions based on a maximization of the value of $W(s,a)$.

\begin{equation} \label{eq:wang}
	W(s,a) = \left\{
\begin{array}{l l}      
    HV(P \cup \overline{r}_{sa}) - dist(\overline{r}^p_{sa}, \overline{r}_{sa} )\hspace{0.5cm} Otherwise \\
    HV(P \cup \overline{r}_{sa}) \hspace{0.5cm} if \hspace{0.25cm} \overline{r}_{sa} \preceq P\\
\end{array}\right.
\end{equation}


\begin{figure}[!t]
\begin{center}
\includegraphics[width=0.65\columnwidth]{img/wangHV}
\end{center}
\caption{$r^p_x$ is the projection of the $r_x$ value on the piecewise linear surface (discontinuous line). The shadowed surface represents $HV(P)$. From~\cite{Weijia2012}.}
\label{fig:wangHV}
\end{figure}
 
The proposed algorithm was employed successfully in two domains: the DST and the Grid Scheduling problem, matching state of the art results in both domains, at the expense of a high computational cost.

As mentioned before, the objective of this paper is to propose an MO-MCTS algorithm that is suitable for real-time domains. Although the work discussed here is influenced by Wang's approach, some modifications need to be done in order to overcome the high computational cost involved by their approach.

In the algorithm proposed in this paper, the vector $\overline{r}$ of rewards that was obtained at the end of an Monte Carlo simulation is back-propagated through the nodes visited in the last iteration until the root is reached. In the vanilla algorithm, each node would use this vector $\overline{r}$ to update its own accumulated reward vector $\overline{R}$. Instead of doing this, each node in the MO-MCTS algorithm keeps a local Pareto front ($P$), updated at each iteration with the reward vector $\overline{r}$ obtained at the end of the simulation. Algorithm~\ref{alg:updPareto} describes how the node statistics are updated.

\begin{algorithm}[!t]
\begin{algorithmic}[1]
\Function{Update}{$node, \overline{r}, dominated=false$}

	\State $node.Visits = node.Visits + 1$
	\State $node.\overline{R} = node.\overline{R} + \overline{r}$
	\If{$!dominated$} 
		\If{$node.P \preceq \overline{r}$}	
			\State $dominated = true$
		\Else
			\State $node.P = node.P \cup \overline{r}$
		\EndIf
	\EndIf

	\State $\Call{Update}{node.parent,\overline{r},dominated}$
	
\EndFunction
\end{algorithmic}
\caption{Pareto MO-MCTS node update.}
\label{alg:updPareto}
\end{algorithm}

In this algorithm, if $\overline{r}$ is not dominated by the local Pareto front, it is added to the front and $\overline{r}$ is propagated to its parent. In case $\overline{r}$ is dominated by the node's Pareto front, the local Pareto front and there is no need to keep this propagation up the tree.

Three observations can be made about the mechanism described here:

\begin{itemize}
\item Each node in the tree has an estimate of the quality of the solutions reachable from there, both as an average (as in the baseline MCTS) and as the best case scenario (by keeping the non-dominated front $P$).
\item By construction, if a reward $\overline{r}$ is dominated by the local front of a node, it is a given that it will be dominated by the nodes above in the tree, so there is no need to update the fronts of the upper nodes, producing little cost in the efficiency of the algorithm.
\item It is easy to infer, from the last point, that the Pareto front of a node cannot be worse than the front of its children (in other words, the front of a child will never dominate that of its parent). Therefore, the root of the tree contains the best non-dominated front ever found during the search.
\end{itemize}

This last detail is important for two main reasons. First of all, the algorithm allows the root to store information indicating which is the action to take in order to converge to any specific solution in the front discovered. This information can be used, when all iterations have been performed, to select the move to perform next. If weights for the different objectives are provided, this weights can be used to select the desired solution in the Pareto front of the root node, and hence select the action that leads to that point. Secondly, the root's Pareto front can be used to measure the global quality of the search using the hypervolume calculation.

Finally, the information stored at each node regarding the local Pareto front can be used to substitute $Q(s,a)$ in the UCB1 equation. The quality of a given pair $(s,a)$ can be obtained by measuring the HV of the Pareto front stored in the node reached from state $s$ after applying action $a$. This can be defined as  $Q(s,a) = HV(P)/N(s)$, and the Upper Confidence Bound equation, referred to here as $MO-UCB$, is described as in Equation~\ref{eq:moucb}). 

\begin{equation}	\label{eq:moucb}
a^* = \argmax_{a \in A(s)} \left\{HV(P)/N(s) + C \sqrt{\frac{ \ln N(s) }{ N(s,a) }}\right\}
\end{equation}

\section{Benchmarks} \label{sec:bench}

\section{Experimentation} \label{sec:exp}

\section{Conclusions and Future Work} \label{sec:conc}




\section*{Acknowledgments}
This work was supported by EPSRC grants EP/H048588/1, under the project entitled ``UCT for Games and Beyond''.


% Can use something like this to put references on a page
% by themselves when using endfloat and the captionsoff option.
\ifCLASSOPTIONcaptionsoff
  \newpage
\fi



% trigger a \newpage just before the given reference
% number - used to balance the columns on the last page
% adjust value as needed - may need to be readjusted if
% the document is modified later
%\IEEEtriggeratref{8}
% The "triggered" command can be changed if desired:
%\IEEEtriggercmd{\enlargethispage{-5in}}

% references section

% can use a bibliography generated by BibTeX as a .bbl file
% BibTeX documentation can be easily obtained at:
% http://www.ctan.org/tex-archive/biblio/bibtex/contrib/doc/
% The IEEEtran BibTeX style support page is at:
% http://www.michaelshell.org/tex/ieeetran/bibtex/
%\bibliographystyle{IEEEtran}
% argument is your BibTeX string definitions and bibliography database(s)
%\bibliography{IEEEabrv,../bib/paper}
%
% <OR> manually copy in the resultant .bbl file
% set second argument of \begin to the number of references
% (used to reserve space for the reference number labels box)

\bibliographystyle{IEEEtran} 
%\bibliography{bibliography,mcts_group_mendeley,mctsbradford_group_mendeley}
\bibliography{biblio}

\begin{IEEEbiography}[{\includegraphics[width=1in,height=1.25in,clip,keepaspectratio]{photos/perez.jpg}}]
{Diego Perez}
received a B.Sc. and a M.Sc. in Computer Science from University Carlos III, Madrid, in 2007. He is currently pursuing a Ph.D. in Artificial Intelligence applied to games at the University of Essex, Colchester. He has published in the domain of Game AI, participated in several Game AI competitions and organized the Physical Travelling Salesman Problem competition, held in IEEE conferences during 2012. He also has programming experience in the videogames industry with titles published for game consoles and PC.
\end{IEEEbiography}


\begin{IEEEbiography}[{\includegraphics[width=1in,height=1.25in,clip,keepaspectratio]{photos/sanaz.jpg}}]
{Sanaz Mostaghim}
BIO HERE.
\end{IEEEbiography}

\begin{IEEEbiography}[{\includegraphics[width=1in,height=1.25in,clip,keepaspectratio]{./photos/spyros.jpg}}]{Spyridon Samothrakis}
 is currently pursuing a PhD in Computational Intelligence and Games at the University of Essex. His interests include game theory, computational neuroscience, evolutionary algorithms and consciousness. 
\end{IEEEbiography}

\begin{biography}[{\includegraphics[width=1in,height=1.25in,clip,keepaspectratio]{photos/lucas.png}}]{Simon Lucas}
(SMIEEE) is a professor of Computer Science at the University of Essex (UK) where he leads the Game Intelligence Group. His main research interests are games, evolutionary computation, and machine learning, and he has published widely in these fields with over 160 peer-reviewed papers. He is the inventor of the scanning n-tuple classifier, and is the founding Editor-in-Chief of the IEEE Transactions on Computational Intelligence and AI in Games.
\end{biography}

% You can push biographies down or up by placing
% a \vfill before or after them. The appropriate
% use of \vfill depends on what kind of text is
% on the last page and whether or not the columns
% are being equalized.

%\vfill

% Can be used to pull up biographies so that the bottom of the last one
% is flush with the other column.
%\enlargethispage{-5in}



% that's all folks
\end{document}
